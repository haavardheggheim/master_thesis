Introduction to the introduction
three pharagrahphs, en dwioth transtition t bntex section
 where full context will be provided


Context
give context 


Restatement of the problem
and siginificance
echo opening but wich much more resonacne for the reader whonow has a deepåer understaning og the reseearch context


Restatement of the response
restaated in  a more meaningduyo detail wo now have a better understanding for the prioblme

Roadmap
biref indicatiionog how the thesis will proceed


effectively exlpain thei research in thrree minutes i9n a language appropriabte to annjo-specialist audioence


\section{Introduction}
We live in a world where communication plays a major role in how we conduct business. Location independant work roles are more common now than ever before. Two persons no longer have to sit next to each other to collaborate on a project. But to be able to collaborate on a project when the persons are not able to communicate directly with each other, the use of good real-time communications technology gets very important. There are many different tools for doing this today, but it's usually been a very complex and difficult business to be involved in.

Huge companies like Cisco and Microsoft have long been the big players in this industry. They have provided tools that makes us able to communicate in real-time. These tools are very expensive and this is why these companies make so much money. But now we see a lot startups showing up competing with the big companies in the communication space. They use new web technologies that simplifies greatly the way we can develop communication applications. These new technologies are integrated in the web browser so that we can use simple web API's to handle communication between peers. These new web APIs are wrapped under a library called \gls{wrtc}. All the advanced underlying technologies of doing transportation of media, routing the traffic through firewalls, and dynamically adjust delivery of media thorugh optimization of networking bandwidth is taken care of by the browser. Using these new simple APIs we can quickly develop advanced communication systems, and since they are integrated directly in the browser, it is also possible to get mobile support with a lot less effort than before. The main problem here is that these new APIs don't have any way of comunicating with older legacy systems. This is a problem for companies that want to take advantage of the new possibilites these API's provide, such as mobile support for almost all platforms that can run a web browser. In collaboration with a company called Visual Solutions I have taken a look at this problem. In this thesis we will provide a model of a solution to this problem. We do this through looking at different models of creating a gateway between \gls{wrtc} and Visual Solutions desktop application for doing collaboration over IP named Virtual Arena.

This thesis will begin by giving a introduction to a \gls{wrtc} and basic transportation protocols. To understand how \gls{wrtc} works you need to understand a lot of new concepts. These will be described in the background chapter. Then we will describe in detail what the main problems of bridging \gls{wrtc} with a desktop application like Virtual Arena are. We will also give a quick overview of what is currently possible to do with these new APIs. We will then model a gateway that makes communication between \gls{wrtc} and Virtual Arena possible. We will also look at how this model can improved by looking at different network architectures other than the simple peer-to-peer architecture.

\gls{wrtc} is a very interesting new technology that will drive a lot of innovation. It allows smaller companies too create communication services much easier than before, and it will allow well established companies to add new features to their current products if using the model provided in this thesis.