%!TEX root = ../main.tex

page 211
state research problem
announce key themes
state the main point or launching point that anticipates the main point

context+problem+main point

put in key search words

\begin{abstract}
% The use of communication technologies today consume a large time of our daily lives. A lot of these technologies are now moving to Internet based solutions. Some of these services are collaboration tools that do video/audio and application sharing. This thesis will look at streaming media on the web in a collaborative enterprise environment. There are currently many Internet based communication services existing today that does this, but most of these are not open, cost money and require the installation of a plugin. This thesis will look at how Visual Solutions can integrate the new WebRTC APIs for doing real-time streaming of media in the browser with their current visual collaboration solution Virtual Arena.

% The aim of this thesis is to find the most suitable way for Visual Solutions to implement live media sharing in the web browser with their current application Virtual Arena. The goal is to detect challenges and problems that might occur with the use of the new Web APIs.

% The HTML5 standard introduces many new APIs that give web browsers the ability to communicate directly with each other in real-time. But there are many solutions and the different providers doesn't seem to agree on a single solution. This thesis will examine using these new APIs to see how we can integrate the browser in current media collaboration systems. The goal being to determine the feasability of using these new APIs and evaluate their usage. This thesis hopes to aid in developing solutions for media sharing in the browser that can be used in an enterprise setting.
\end{abstract}


This is a study on how to integrate \gls{rtcweb} with a enterprise communication system. Each system is broken down into several components on the transport level. A gateway model interoperating between the two architectures are created and then evaluated by looking at similar systems.

\\gls{rtcweb} is defined as the underlying technologies of a set of APIs drafted by the \gls{w3c} as \gls{webrtc}. It supports browser-to-browser applications for audio/video chatting without plugins. \gls{wrtc} is currently seeing great support with two of the major browser vendors; namely Google and Mozilla. Both Apple and Microsoft currently lag a bit behind, but they are both active in the working groups that define the standards. A ton of new companies are appearing using this new technology because of the easy entry these APIs provide. The browser APIs are simple to use and let's you basically create your own phone company over a weekend. This has turned the eye of bigger telecom vendors like Cisco and Ericsson. All of them are looking into how they can use \gls{wrtc} in the future. One of the greatest advantage of utilizing \gls{wrtc},% is the whole range of new devices you suddenly have access to. Using the optimized audio and video engine of \gls{rtcweb}, you can scale for pretty much any device that has support for the latest Chrome or Firefox browser. The integration of \gls{wrtc} allows for easier adopting of the \gls{byod} to work policy that permits employees to bring their personally owned mobile devices (laptops, tablets, and smart phones) to work. The problems of using a mobile \gls{wrtc} client within an enterprise setting is discussed in this thesis.  

Using guidelines derived from looking at different ways of doing this integration, in order to aid in deciding if this is a feasible way to go about getting support for mobile devices in current enterprise systems. The use of a gateway outlined in this thesis is the main topic together with recommendations for implementing a mobile \gls{wrtc} client for tablets and smartphones.