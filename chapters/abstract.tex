%!TEX root = ../main.tex

\begin{abstract}
% The use of communication technologies today consume a large time of our daily lives. A lot of these technologies are now moving to Internet based solutions. Some of these services are collaboration tools that do video/audio and application sharing. This thesis will look at streaming media on the web in a collaborative enterprise environment. There are currently many Internet based communication services existing today that does this, but most of these are not open, cost money and require the installation of a plugin. This thesis will look at how Visual Solutions can integrate the new WebRTC APIs for doing real-time streaming of media in the browser with their current visual collaboration solution Virtual Arena.

% The aim of this thesis is to find the most suitable way for Visual Solutions to implement live media sharing in the web browser with their current application Virtual Arena. The goal is to detect challenges and problems that might occur with the use of the new Web APIs.

% The HTML5 standard introduces many new APIs that give web browsers the ability to communicate directly with each other in real-time. But there are many solutions and the different providers doesn't seem to agree on a single solution. This thesis will examine using these new APIs to see how we can integrate the browser in current media collaboration systems. The goal being to determine the feasability of using these new APIs and evaluate their usage. This thesis hopes to aid in developing solutions for media sharing in the browser that can be used in an enterprise setting.
\end{abstract}