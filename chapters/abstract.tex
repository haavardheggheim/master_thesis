%!TEX root = ../main.tex

% page 211
% state research problem
% announce key themes
% state the main point or launching point that anticipates the main point

% context+problem+main point

% put in key search words

% \begin{abstract}
% TODO
% % The use of communication technologies today consume a large time of our daily lives. A lot of these technologies are now moving to Internet based solutions. Some of these services are collaboration tools that do video/audio and application sharing. This thesis will look at streaming media on the web in a collaborative enterprise environment. There are currently many Internet based communication services existing today that does this, but most of these are not open, cost money and require the installation of a plugin. This thesis will look at how Visual Solutions can integrate the new WebRTC APIs for doing real-time streaming of media in the browser with their current visual collaboration solution Virtual Arena.

% % The aim of this thesis is to find the most suitable way for Visual Solutions to implement live media sharing in the web browser with their current application Virtual Arena. The goal is to detect challenges and problems that might occur with the use of the new Web APIs.

% % The HTML5 standard introduces many new APIs that give web browsers the ability to communicate directly with each other in real-time. But there are many solutions and the different providers doesn't seem to agree on a single solution. This thesis will examine using these new APIs to see how we can integrate the browser in current media collaboration systems. The goal being to determine the feasability of using these new APIs and evaluate their usage. This thesis hopes to aid in developing solutions for media sharing in the browser that can be used in an enterprise setting.
% \end{abstract}

\begin{abstract}
This is a study on how to integrate \gls{wrtc} with an enterprise collaboration system. This is done by modeling a gateway that translates all the traffic between the two endpoints. By breaking down the parts of the interaction into subcomponents, this study explains the technologies involved and derives guidelines for implementing the gateway.

WebRTC consists of a set of Javascript APIs drafted by the \gls{w3c}. It allows for browser-to-browser communication of audio and video without plugins. It uses \gls{rtcweb} for the media stack, which allows for peer-to-peer (P2P) communication between browsers. One of the great advantages of utilizing WebRTC is the whole range of new devices you suddenly can communicate with, because it runs directly in the browser. Using the optimized audio and video engine of RTCWeb, you can scale for pretty much any device that has support for the latest Chrome or Firefox browser. The integration of \gls{wrtc} allows for easier adopting of the \gls{byod} to work policy that permits employees to bring their personally owned mobile devices (laptops, tablets, and smart phones) to work. The issues of developing a WebRTC client for mobile \gls{wrtc} devices are also discussed in this thesis. 

There already exists gateways today that does this work, but none of them addresses problems that enterprises have to acknowledge when implementing such systems, such as crossing the enterprise firewall, identifying and authorizing users. This thesis derives guidelines for integrating WebRTC with enterprise systems for doing communication, with the addition of addressing issues that occur when developing WebRTC clients for mobile devices. The modeling of a gateway in this thesis is the main topic together with recommendations for implementing a mobile WebRTC client for tablets and smartphones.
\end{abstract}