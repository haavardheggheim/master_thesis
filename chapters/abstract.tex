%!TEX root = ../main.tex

% page 211
% state research problem
% announce key themes
% state the main point or launching point that anticipates the main point

% context+problem+main point

% put in key search words

\begin{abstract}
This is a study on how to integrate \gls{wrtc} with enterprise collaboration systems. This is done by modeling a gateway that translates all the traffic between a client within an enterprise environment and a web client. The traffic has to cross a firewall with strict access policies to finally reach its destination. By breaking down the parts of the interaction into subcomponents, this study explains the technologies involved and derives guidelines for implementing the gateway.  

WebRTC consists of a set of Javascript APIs drafted by the \gls{w3c}. It allows for browser-to-browser communication of audio and video without plugins. It uses \gls{rtcweb} for the media stack. One of the great advantages of utilizing WebRTC is the whole range of new devices you can communicate with, since the technology runs directly in the browser. Using the optimized audio and video engine of RTCWeb, you can adapt for pretty much any device that has support for the latest versions of Chrome or Firefox. The integration of \gls{wrtc} allows for easier adopting of the \gls{byod} to work policy that permits employees to bring their personally owned mobile devices (laptops, tablets, and smart phones) to work. Issues around developing a WebRTC client for mobile devices are also discussed in this thesis. 

My original contribution in this field is to derive guidelines for integrating WebRTC with enterprise systems for doing communication, with the addition of addressing issues that occur when developing WebRTC clients for mobile devices.
\end{abstract}