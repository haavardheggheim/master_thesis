%!TEX root = ../main.tex

This chapter will explain the background for this thesis, which is divided into sections. I start with the introduction of Visual Solutions and their practical problem. This section will also give a quick overview of how the application Virtual Arena works by describing its underlying architecture. Secondly, I will introduce the RTCWeb technologies and WebRTC browser APIs. Then I will look at the use of \gls{wrtc} on mobile devices. Finally, I will look at competing real-time communication vendors and similar work.

\section{Visual Solutions}
This section will first introduce the company Visual Solutions who provided the practical problem of integrating \gls{wrtc} with their own application for doing collaboration. Secondly I will give a quick overview of how the application Virtual Arena works by describing its underlying architecture, then I will look at some of the security concerns that occur when an enterprise collaboration system has to access the public internet.

\subsection{Visual Solutions}
Visual Solutions is a Norwegian company in the BB Visual Group\footnote{http://www.bbvisualgroup.com/}. Their primary business is within the integration of operations for the oil and gas industry. They create solutions enabling collaboration across organisation units and geographic locations. One of their applications Virtual Arena\cite{solutions_b2_virtual_2014}, which from now on will be referred to as VA, is a powerful and interactive tool that allows for high-performance application sharing, with audio and video communication from a 3D shared scene as seen in Figure \ref{fig:vsva-3d-scene}. VA supports many-to-many collaborative scenarios by utilizing a media server which will be described in the next section.
\\
\begin{figure}[here]
\centerline{\includegraphics[scale=0.6]{virtualarena.png}}
\caption{Virtual Arena 3D shared scene}
\label{fig:vsva-3d-scene}
\end{figure}

\subsection{Virtual Arena}
VA is the application that Visual Solutions has created for doing visual collaboration over IP. The architecture of VA is visualized in Figure \ref{fig:vsva-architecture}. The application uses a media server that serves multiple purposes. It acts as a \gls{mcu}, applies mitigation strategies for scenarios with limited bandwidth, and also sharing of applicaton data. Mitigation strategies can f.ex be reducing the video bitrate to adjust and adapt for poor connections. The media server works together with a router for distributing the streams. By utilizing a media server VA can support a lot of incoming and outgoing streams. A client can subscribe to multiple streams of audio/video and applications. It manages these connections using a tree structure, but this is an advanced topic that I won't go into detail about. In the next subsections the different parts of the architecture in Figure \ref{fig:vsva-architecture} will be described with the necessary information required to understand the practical problem this thesis will try to solve.

\begin{figure}[here]
\centerline{\includegraphics[scale=0.6]{enterprise-mcu-architecture.png}}
\caption{Virtual Arena systems architecture}
\label{fig:vsva-architecture}
\end{figure}

\subsubsection{Signaling}
VA has a proprietary way of doing signaling over \gls{rtcp}. RTCP is a control protocol for \gls{rtp}. This is where the clients send messages to the media server describing which streams they want to subscribe to. Communication between peers and the media server is done by opening up ports in the firewall to listen for incoming TCP and UDP connections. These ports are preconfigured for this application. The media server can receive incoming streams and route it to all peers connected. The streams are identified using a \gls{ssrc}, which is defined in the header packets of the RTP streams.

\subsubsection{Transport}
VA uses raw RTP streams over UDP. This is basically the standard protocol used for transmitting real-time media in any communication system. There will be more details about this protocol in the upcoming sections.

\subsubsection{Media}
VA uses patent-free codecs for audio and video. It uses Speex for audio, and Theora for video. The Speex codec is designed for high quality speech and low bitrate, which is perfect for audio communications. It supports both narrowband and wideband sampling rates in the same bit-stream\cite{speex}, which allows for minimizing bandwidth and an adaptable sound quality. The Theora codec is a common choice in enterprise systems, it is less CPU-intensive than the popular H.264 codec\cite{theora}, which is licensed by Cisco. Theora is also license free, however H.264 offers the added benefits of hardware acceleration in a lot of graphic cards, and Cisco has recently announced that they will make H.264 open-source\cite{h264-free}, which will effectively make the codec free to use in technologies like \gls{wrtc}.

\subsection{Security}
VA only uses raw RTP streams, because no security is needed. It operates in a closed business environement, so transport level encryption is not necessary, because unidentified peers are not allowed inside the network anyways. The enterprise firewall has very strict polices, only allowing certain kinds of traffic on specific ports. VA enforce this by using a DMZ??? It works like this....

\subsection*{Summary}
VA operates very much like a typical enterprise comunication system. It uses common transport protocols and codecs, however it's architecture is still very different from the protocols and codecs defined in \gls{rtcweb}, which we will see in the next section.

\newpage
\section{Introduction to WebRTC}

\gls{wrtc} is a collection of standards, protocols, and JavaScript APIs. The combination of these enables web browsers to do peer-to-peer audio, video and data sharing between browsers. There is no plugin or third-party software required. Real-time communication is now becoming a standard feature in browsers that any web site can use via simple APIs. Delivering functionality such as live audio and video sharing and data exchange requires a lot of new processing capabilities in the browser. I will look at the underlying technologies and protocols of \gls{wrtc}, such as audio and video processing, transportation of media, and security. These technologies are abstracted behind three primary browser APIs:

\begin{itemize}
\item MediaStream: capturing audio and video streams
\item RTCPeerConnection: communication of audio and video data
\item RTCDataChannel: communication of arbitrary application data
\end{itemize}

With the above APIs you can: capture media from a camera on your device, do peer discovery, connection negotiations, real-time transportation of media and sharing of arbitrary data. But first of all we look at the working groups behind the development of \gls{wrtc}.

\subsection{Standards and development}
The \gls{wrtc} architecture consists of different standards, covering both protocols and browser APIs:

\begin{itemize}
\item All the different protocols, data formats, and security required to make \gls{wrtc} work, are defined by an IETF working group. The protocol suite is called \gls{rtcweb}.
\item The components used for setting up communication sessions and accessing the camera and microphone, are defined by a \gls{w3c} working group. They are responsible for drafting the WebRTC browser API definitions.
\end{itemize}

WebRTC is the first open standard to transport data over UDP in the browser. However doing this requires a lot more than raw UDP to do real-time communication, we need media processing and added security layers.

\subsection{Audio and video}
Doing live audio and video sharing requires processing to enhance image quality, doing synchronization, echo cancellation, noise reduction and packet loss concealment\cite{grigorik_high_2013}. On the transmitting end the bitrate must be adjusted to fluctuating bandwidth and latency between clients. On the receiving end the client must decode the streams in real-time and be able to adjust network jitter and latency delays. These are complex problems, but WebRTC includes fully featured audio and video engines in the browser as seen in Figure \ref{fig:audiovideocapture}, which takes care of the signal processing for us. All of the processing is done directly by the browser.

\begin{figure}[here]
\centerline{\includegraphics[scale=0.8]{audiovideocapture.png}}
\caption{WebRTC audio and video engines. Reprinted from: High Performance Browser Networking\cite{grigorik_high_2013}.}
\label{fig:audiovideocapture}
\end{figure}

\subsection{Real-time transports}
When it comes to real-time communication, synchronization and low latency is more important than reliability. This is the reason why the UDP protocol is preferred for doing real-time communication. While TCP delivers a reliable communication, there can be delays. If a packet is lost, it is re-requested. The human brain doesn't handle latency in communication very well, but we are good at filling in the gaps. Therefore we use UDP, which is a connectionless solution. It doesn't check the state of the message. So part of the message could be lost, and we wouldn't know, but the connection would run without delay.

UDP is the foundation for doing real-time communication, but to meet all the specified requirements of \gls{wrtc}, we need to support a lot of protocols and services on top of that as we can see in Figure \ref{fig:wrtc-protocol-stack}.

\begin{figure}[here]
\centerline{\includegraphics[scale=0.8]{wrtc-protocol-stack.png}}
\caption{WebRTC protocol stack. Reprinted from: High Performance Browser Networking\cite{grigorik_high_2013}.}
\label{fig:wrtc-protocol-stack}
\end{figure}

\gls{ice}, \gls{stun}, and \gls{turn} are needed to establish a connection over UDP in \gls{wrtc}. Encryption is mandatory and \gls{dtls} is used to secure all transfers between peers. \gls{srtp} and \gls{sctp} are used to multiplex the different streams, provide congestion and flow control, and provide delivery on top of UDP.

After looking at the underlying technologies of \gls{wrtc} defined as RTCWeb, we will now give an introduction to the drafted browser APIs. We are only going to introduce the first two of them, since the third one is not applicable to this project.

\subsection{MediaStream}
Acquiring audio and video is done by calling the getUserMedia() method that enables the browser to acquire audio and video from a physical device such as a webcam or microphone. Incoming streams from remote network peers are also captured and everything is packaged in a MediaStream object. Inside the MediaStream object we have one or more individual tracks that are synchronized with one another. The output can be sent to a local audio or video element, post-processing scripts or remote peers.

The MediaStream object represents a real-time media stream and allows the application to manipulate individual tracks and specify outputs.

The getUserMedia() method allows us to specify a list of mandatory constraints to match the needs of the application as seen in Listing \ref{lst:constraints}.

\lstset{language=Javascript} 
\begin{lstlisting}[caption={getUserMedia constraints object}, label={lst:constraints}]
var constraints = {
	audio: true,
	video: {
		mandatory: {
			width: { min: 1280 },
			height: { min: 720 },
			frameRate: 30
		},
		optional: []
	}
}

navigator.getUserMedia(constraints, stream, error);
\end{lstlisting}

Once a stream is acquired we can feed them into other APIs such as Web Audio for enabling advanced audio processing. Canvas API for post-processing video frames and WebGL can apply 3D effects on the output stream.

Simplified the getUserMedia() is a method to acquire audio and video streams. The media is automatically optimized, encoded and decoded by the audio and video engines. Then we can display the media locally in an audio or video element in the browser.


\subsection{RTCPeerConnection}
The RTCPeerConnection as seen in Figure \ref{fig:rtcpeerconnection} is responsible for managing the peer-to-peer connection. It uses an \gls{ice} Agent for NAT traversal, keeps track of streams, and triggers renegotiation when required. It provides an API for generating offer and answer.

To be able to understand RTCPeerConnection we need to understand signaling and \gls{ice}.

\begin{figure}[here]
\centerline{\includegraphics[scale=0.8]{peerconnection.png}}
\caption{RTCPeerConnection API. Reprinted from: High Performance Browser Networking\cite{grigorik_high_2013}.}
\label{fig:rtcpeerconnection}
\end{figure}

\subsubsection{Establishing a connection}

\begin{enumerate}
\item Input devices are opened for capture as the media source. This is done using the getUserMedia() method.
\item Now we have to signal the other users that we want to connect to them. Using RTCPeerConnection we send a \gls{sdp} offer to the other clients as seen in Figure \ref{fig:sdp-exchange}, which generates an \gls{sdp}. The \gls{sdp} here includes \gls{ice} candidates,  which allows for firewall traversal.

\begin{figure}[here]
\centerline{\includegraphics[scale=1.0]{SDPexchange.png}}
\caption{Offer/answer SDP exchange between peers. Reprinted from: High Performance Browser Networking\cite{grigorik_high_2013}.}
\label{fig:sdp-exchange}
\end{figure}

\item Once connection is successful, a \gls{dtls} connection is opened and all the media from input devices are encoded into packets and transmitted using \gls{srtp}-\gls{dtls} streams.
\item At the destination, the packets are decoded and formatted into a MediaStream.
\item The MediaStream is sent to output devices
\end{enumerate}


\subsubsection{Signaling}
While \gls{wrtc} does all the routing and connectivity check for us with the \gls{ice} protocol, we have to do session negotiation ourselves. To do this we must extend an offer to the receiving peer and we need an answer in return. Choice of signaling application is up to us. The \gls{wrtc} standard does not define a signaling protocol, but the key information that needs to be exchanged is the \gls{sdp}, which specifies the necessary transport and media configuration information necessary to establish a connection. This approach is outlined by \gls{jsep}. Assuming we have a shared signaling channel, we can initiate a \gls{wrtc} connection (Listing \ref{lst:peerConnection}).


\begin{lstlisting}[caption={Initiating a RTCPeerConnection in WebRTC}, label={lst:peerConnection}]
var signalingChannel = new SignalingChannel();
var pc = new RTCPeerConnection({});

navigator.getUserMedia(constraints, onStream, error);

function onStream(stream) {
  pc.addstream(stream);

  pc.createOffer(function(offer) {
    pc.setLocalDescription(offer);
    signalingChannel.send(offer.sdp);
  });
}
\end{lstlisting}


\subsubsection{SDP}
\gls{wrtc} uses a \gls{sdp} to describe the parameters of a connection. It represents a list of properties describing the connection, ICE candidates, \gls{dtls} parameters, types of media, codecs, bandwidth, \gls{ssrc}s and other metadata information\footnote{http://tools.ietf.org/id/draft-nandakumar-rtcweb-sdp-01.html}.



Here is some of the information that is generated after a call createOffer() has generated the \gls{sdp} description (Listing \ref{lst:sdp}):

\begin{lstlisting}[frame=single, caption={Snippet of a SDP description}, label={lst:sdp}]
...snip...
m=audio 1 RTP/SAVPF 111 103 104 0 8 106 105 13 126
c=IN IP4 0.0.0.0
a=rtcp:1 IN IP4 0.0.0.0
a=ice-ufrag:fAYfQM/iWMQPqiHs
a=ice-pwd:pgbuPPRdpKq+obC0lyRxVDe/
a=extmap:1 urn:ietf:params:rtp-hdrext:ssrc-audio-level
a=rtpmap:111 opus/48000/2
a=maxptime:60
a=ssrc:2209464108 cname:7oIEPieg3XZzHJdN
a=ssrc:2209464108 mslabel:uWu6kVvHhYbbkOtNalf5E2LFgjx4cpGMhnfo
a=ssrc:2209464108 label:2b626a18-c54c-4c1b-9f42-03519a9b63f2
m=video 1 RTP/SAVPF 100 116 117
...snip...
\end{lstlisting}


\subsubsection{ICE}
\label{sec:ice}
In order to establish a peer-to-peer connection, the peers must be able to send packets to each other. This is easy when you know which IP and port to listen to for incoming messages, but hard when you don't know. Normally there would be firewalls and NAT devices between most peers. In a local environment where there is no firewall, we could establish a connection between two peers by appending the IP and port number to the \gls{sdp}, and forward it to the other peer. What \gls{ice} does is getting around these restrictions by doing connectivity checks and route planning between peers.

\gls{ice} gathers all possible addresses it can in address:port and transport triplets\cite{ivov_ice_2013}. \gls{ice} calls these `candidates', and once candidates have been gathered, they are ordered in a list based on priority. Highest priorities are assigned to candidates with the least overhead: those that you get from the device itself, the IP `host' candidates. Next are STUN candidates, which are f.ex obtained via \gls{upnp}. Last option is the `relayed' candidates that are obtained from TURN servers. By relaying media through a TURN server, we no longer have a peer-to-peer connection, this is costly and not something we want to do unless there is no other way of connecting to each other.

\subsubsection{Secure communication}
Once we have completed the \gls{sdp} answers and offers, and traversed NATs, we have come a long way. But \gls{wrtc} require that we encrypt all communication. On top of UDP, we have \gls{srtp} used for transporting media securely, and \gls{dtls} which is used to negotiate secret keys for encrypting media data. This is all taken care of by \gls{wrtc}, and once we have everything else in place, we are ready to establish peer-to-peer connections.

\subsubsection{Bringing it all together}
To summarize the process of creating a peer-to-peer connection (Listing \ref{lst:p2p-webrtc-1}, \ref{lst:p2p-webrtc-2} and \ref{lst:p2p-webrtc-3}):

\begin{lstlisting}[caption={Initializing a WebRTC P2P connection, Part I}, label={lst:p2p-webrtc-1}]
<video id='local' autoplay></video>
<video id='remote' autoplay></video>

<var ice = {"iceServers": [
    {"url": "stun:stunserver.com:12345"},
    {"url": "turn:turnserver.com", "username": "user", "credential": "pass"}
  ]};

  var signalingChannel = new SignalingChannel();
  var pc = new RTCPeerConnection(ice);

  navigator.getUserMedia({ "audio": true, "video": true }, onStream, logError);
\end{lstlisting}

In Listing \ref{lst:p2p-webrtc-1} we have defined the output elements for both local and remote streams. We have initialized a signaling channel and peer connection object. Lastly, we have used the getUserMedia() method for acquiring local audio and video streams.

\begin{lstlisting}[caption={Initializing a WebRTC P2P connection, Part II}, label={lst:p2p-webrtc-2}]
  function onStream(evt) {
    pc.addStream(evt.stream);

    var localVideo = document.getElementById('localVideo');
    localVideo.src = window.URL.createObjectURL(evt.stream);

    pc.createOffer(function(offer) {
      pc.setLocalDescription(offer);
      signalingChannel.send(offer.sdp);
    });
  }

  pc.onicecandidate = function(evt) {
    if (evt.candidate) {
      signalingChannel.send(evt.candidate);
    }
  }
\end{lstlisting}

In Listing \ref{lst:p2p-webrtc-2} we register the local media streams with PeerConnection. Then we output the local video stream to its video element. After that we generate a SDP offer describing the connection and gather ICE candidates. Then we send it to the other peer via the signaling channel.

\begin{lstlisting}[caption={Initializing a WebRTC P2P connection, Part III}, label={lst:p2p-webrtc-3}]
  signalingChannel.onmessage = function(msg) {
    if (msg.candidate) {
      pc.addIceCandidate(msg.candidate);
    }
  }

  pc.onaddstream = function (evt) {
    var remoteVideo = document.getElementById('remoteVideo');
    remoteVideo.src = window.URL.createObjectURL(evt.stream);
  }

  function logError() { ... }
</script>
\end{lstlisting}

Later in Listing \ref{lst:p2p-webrtc-3} we register the remote ICE candidates and initiate the connection. Lastly, we receive the remote video stream and output to the video element.

\subsection{Summary}
We have now looked at the underlying technologies of \gls{wrtc} and the subsequent browser APIs. We know that there are two different aspects of this technology: the RTCWeb protocols and the WebRTC browser APIs. We also know how to initiate a simple peer-to-peer connection.

\newpage
\section{Mobile devices}
%!TEX root = ../main.tex

The use of mobile broadband for accessing the Internet via tablets and smartphones has increased rapidly as a result of high speed mobile networks, such as 3G and 4G. The web browsers on these devices are becoming more and more similar in capabilities to their desktop versions. It is possible to run \gls{wrtc} applications in some of them, and these features will become more available with time. This section is going to look at the use of \gls{wrtc} in mobile devices. We start by looking at some of the issues we have to deal with when working with mobile devices, then we will look at performance and quality metrics derived from current research.

\subsection{Network performance}
It is feasible to run \gls{wrtc} applications on mobile devices. However, battery consumption, persistent connectivity, and quality performance remain a big challenge. A lot of work is being done in radio technologies, but the most important factors are how the applications, protocols, and API's are designed\cite{isomaki2012considerations}. First we will look at some of the performance issues of serving wireless video.

The most used and relevant networks are Wi-Fi, WiMAX, and the different telecommunication networks. Serving wireless video is a significant challenge for already-stressed cellular data networks\cite{erman2011over}. Video traffic requires high bandwidth, imposes latency and packet loss. The key is to deliver video as high in quality as the network supports. By looking at a study testing video and audio quality on two differenet WiMAX networks\cite{fund2013performance}, we can see variations of performance in two different locations.

\pagebreak
\begin{figure}[here]
\centerline{\includegraphics[width=\textwidth]{mobile-video-metrics.png}}
\caption{WebRTC video performance. Adapted from: Performance of DASH and WebRTC Video Services for Mobile Users\cite{fund2013performance}.}
\label{fig:mobile-video-metrics}
\end{figure}

\gls{wrtc} adapts to changing link conditions by estimating the available sent bandwidth and passing this estimate to the encoder as a target bitrate. Figure \ref{fig:mobile-video-metrics} show key metrics of video and audio quality for a \gls{wrtc} session sustained over a WiMAX link in two different locations. The color blue is a suburban setting (Amherst) and purple is an urban setting (New York). For each plot, the shaded area shows one standard deviation above and below the mean, while the points give values for a representative trial. These values are presented from the perspective of the mobile WiMAX-connected peer. There is a dramatic difference in performance between the two locations, with the suburban setting always outperforming the urban setting. The uplink stream has significantly worse video performance than the downlink. This is probably due to the asymmetry of the cellular link. Only 25\% of radio resources are allocated to the uplink\cite{fund2013performance}. A result of this is a reduced frame rate, this is unfortunate since mobile users are often walking, which makes the video feed have a high motion content.

\pagebreak
\begin{figure}[here]
\centerline{\includegraphics[scale=0.55]{mobile-audio-metrics.png}}
\caption{WebRTC audio performance. Reprinted from: Performance of DASH and WebRTC Video Services for Mobile Users\cite{fund2013performance}.}
\label{fig:mobile-audio-metrics}
\end{figure}

An important thing to mention is that video quality is not that important when it comes to communication, the audio quality is of much more value\cite{fund2013performance}. In Figure \ref{fig:mobile-audio-metrics} audio performance is shown as representative values (points) and a range of one standard deviation above and below the mean (shaded area). We can observe that the audio bitrate is mostly consistent, degrading only in extremely poor channel quality. The biggest problem here as seen in Table \ref{tbl:wrtc-packet-loss} is that the packet loss is quite high on the uplink for audio in the urban environment. This is because the quality is not reduced to keep the audio signal clear. This could be somewhat fixed by retransmission of the packets, but this will again increse latency.
\\
\begin{table}[here]
\centerline{\includegraphics[scale=0.5]{wrtc-packet-loss}}
\caption{Packet loss ratio for a WebRTC session. Adapted from: Performance of DASH and WebRTC Video Services for Mobile Users\cite{fund2013performance}.}
\label{tbl:wrtc-packet-loss}
\end{table}

\subsection{Connectivity}
Another concern is persistent connectivity. For an app to be reachable for incoming connections, it needs some kind of persistant communication channel. Most cellular networks have a firewall preventing incoming TCP connections\cite{isomaki2012considerations}. To keep a TCP connection alive, the application needs to send some kind of keep-alive packets with a high enough frequency to avoid a connection timeout. Problem with keeping a connection open all the time, is that the device's radio will consume a lot of battery. %herefore, the keep-alive packets should only be sent infrequently.

\subsection{Summary}
Most of the performance issues is handled by the underlying technologies of \gls{wrtc}, which actually is pretty good at adapting bandwidth usage in variable network conditions. But it is important to be aware of these problems, when developing applications. The effect that a mobile users uplink is much worse than it's downlink has great impact on a user's perception of the connection. However, the most important factor for us to consider is how we will do the signaling with a mobile device, since we can't keep a connection open all the time, as that would drain the battery.

\newpage
\section{Enterprise communication vendors}
\input{chapters/enterprise-communication-vendors.tex}

\newpage
\section{Similar work}
In this section I will look at similar work that is relevant for my thesis. These works are not finished, so they are subject to change at any time.

\subsection{WebRTC to IP Multimedia Subsystem}
\label{sec:webrtc-ims}
There is an ongoing effort at 3GPP on giving \gls{wrtc} clients access to \gls{ims} with 3GPP TR 23.701\cite{3gpp-wrtc-access-ims} as the current proposed architecture. \gls{ims} is basically an architectural framework for delivering IP multimedia services. It was originally designed for evolving mobile networks beyond \gls{gsm}. Later it has evolved to include Wireless LAN and fixed lines, it's intended to aid the access of multimedia and voice applications across different networks. This study is relevant because most telecom companies already have IMS or a similar architecture in place for doing real-time communication. The following diagram shows the proposed architecture:

\begin{figure}[here]
\centerline{\includegraphics[width=\textwidth]{3gpp-wrtc-ims-architecture.png}}
\caption{WebRTC IMS architecture by 3GPP TR 23.701.}
\label{fig:wrtc-ims-architecture}
\end{figure}

Let's look at the main elements:

\textbf{WebRTC IMS Client (WIC)}
is downloaded from the WWSF seen in Figure \ref{fig:wrtc-ims-architecture}. It provides the application logic and \gls{wrtc} API calls to access the communication services of \gls{ims}. It works on any device supporting a WebRTC capable browser.

\textbf{User Equipment(UE)}
is a device or application. It can be a web application running in a browser, a tablet or mobile phone.

\textbf{WebRTC Web Server Function (WWSF)}
is from where the client is downloaded. This could be a web server hosting the WIC, or an app store such as Google Play.

\textbf{P-CSCF enhanced for WebRTC (eP-CSCF)}
is the entry point of \gls{sip} requests. SIP is a protocol used for signaling, and they propose to receive SIP over WebSockets. It adapts signaling on the \gls{wrtc} side to standard IMS-SIP. The specification is open to the use of different protocols, but this is the proposed solution.

\textbf{IMS Access Gateway enhanced for WebRTC (eIMS-AGW)}
supports \gls{rtcweb} media as defined by \gls{ietf}. It needs to support these functions:
\begin{itemize}
\item{SRTP-DTLS. \gls{sdes} is what's used in \gls{ims}}
\item{Audio/video transcoding. H.264 is the \gls{ims} standard}
\item{RTCP demultiplexing. \gls{rtcweb} supports multiplexing of audio/video calls and RTP/RTCP over same RTP session and port. This is not supported in \gls{ims}, so this component needs to support demultiplexing.}
\end{itemize}

In addition this component must also support STUN connectivity checks.

\textbf{IP Connectivity Access Network (IP-CAN)}
is used to reach the IMS core from the UE. Can be LTE\footnote{Long-Term Evolution is a standard for wireless communication of high-speed data for mobile phones.} for mobile, DSL or WLAN.

\textbf{Policy and Charging Rules Function (PCRF) and Policy and Charging Enforcement Function (PCEF)}
supports policy and charging control decisions based on session and media-related information obtained from the P-CSCF. It uses deep packet inspection and decides based on rules whether the traffic is allowed or not.

\textbf{Network Address Translation (NAT)}
the WIC would normally be behind a NAT element, so a box has been included in the diagram.

\subsection{The webrtc2sip Smart SIP and Media Gateway}
\label{sec:webrtc2sip}
Doubango Telecom has created a gateway to connect WebRTC with SIP endpoints\footnote{http://webrtc2sip.org/}. It's called webrtc2sip and it uses RTCWeb and SIP technologies. It allows your browser to make and receive calls to and from any SIP endpoint. The gateway contains four modules: a SIP Proxy, a RTCWeb Breaker, a Media Coder, and a Click-to-Call service. The first three are relevant to integrating WebRTC with enterprise communications. The SIP Proxy converts the SIP transport from WebSockets to SIP over UDP or TCP, which is supported by most enterprise networks, and the RTCWeb Breaker make support for ICE and SRTP-DTLS when negotiating media streams between WebRTC and SIP endpoints. Lastly, the Media Coder will allow us to make calls between browsers and SIP endpoints supporting different codecs.

%%flere paper's

\subsection{Other relevant papers}
Ther are a few papers relevant to this work such as `Taking on WebRTC in an Enterprise\cite{johnston_taking_2013}', which discusses issues around crossing the enterprise firewall. Another paper that is similar to section~\ref{sec:webrtc-ims} is `Research On The Integration of RTCWeb Technology with IP Multimedia Subsystem\cite{li2012research}'.

\newpage
\section{Background summary}
\gls{wrtc} provides the foundation to interoperate different systems more easily by standardizing real-time web communications. It is an effective tool that can be used to quickly create audio and video communication applications, with the added advantage of support for a variety of mobile devices. With the new APIs we can run a client on any tablet or mobile phone that has a web browser with support for the \gls{wrtc} standard. Related technologies developed by Cisco and Microsoft are more complete and are now dominating the communication space, but it's going to be interesting to see if they will lose some of their market share to smaller companies utilizing \gls{wrtc} in the near future. Some similar work has already been done, such as the webrtc2sip gateway created by Doubango Telecom.

Visual Solutions VA application is using transport technologies that need to have added security layers to work with \gls{wrtc}, and the media formats needs to be transcoded with the appropriate codecs. Also, signaling and routing needs to be enhanced to make VA work with \gls{wrtc}. In the next chapter I will describe the different problems that needs to be addressed.

%Google has successfully used \gls{wrtc} in their Hangout application and on their Chromecast devices. 