In this chapter we will explore the technical requirements for integrating \gls{rtcweb} with a typical enterprise communication system architecture and give the suggested solution. Then we will analyze some of the problems raised by the integration.

\section{Integration}
There are basically two ways of going about this problem. The first would be to directly integrate \gls{wrtc} in the enterprise architecture. We could then use SIP-over-WebSocket signaling and use an end-to-end media path enabled by \gls{wrtc}. This could work if the legacy enterprise system supported all the required protocols defined in \gls{wrtc}, otherwise the architecture would have to be upgraded to support all the new features. This is not trivial, since it would require to redo all of the existing architecture just to support \gls{wrtc}. It is better to create a bridge in the form of a gateway between \gls{wrtc} and the legacy architecture.

\subsection{The Gateway}
First showing a simple web application P2P WebRTC structure. Here signaling goes through a signaling server and media travels directly between the two peers(browsers). A STUN server is used for receiving ICE candidates, if that fails there is a fallback transporting media through a TURN server.

\begin{figure}[here]
\centerline{\includegraphics[scale=0.40]{webrtc-p2p-architecture.png}}
\caption{WebRTC P2P architecture}
\label{fig:jsep}
\end{figure}

Secondly showing a typical enterprise architecture with a signaling server, media server, and a server for routing the streams to peers.

\begin{figure}[here]
\centerline{\includegraphics[scale=0.6]{enterprise-mcu-architecture.png}}
\caption{Enterprise communication platform architecture}
\label{fig:VirtualArenaArchitecture}
\end{figure}

Let's see how RTCWeb can be used to connect to existing enterprise communication networks. In the solution figure we add an interworking gateway component. This component includes three subcomponents:

\begin{itemize}
\item{Signaling Proxy}
\item{Media Transport Proxy}
\item{Media Transcoder}
\end{itemize}

These components should provide all the necessary functionality we need.

\begin{figure}[here]
\centerline{\includegraphics[scale=0.40]{gateway2.png}}
\caption{WebRTC-Enterprise Gateway}
\label{fig:wrtc-enterprise-gateway}
\end{figure}

\subsection{The Signaling Proxy}
In this implemenation the client has implemented signaling using a \gls{sip} stack in Javascript. The SIP includes the \gls{sdp} in this is transmitted over \gls{http} using WebSockets. Then the signaling proxy will manipulate the \gls{sdp} to support the enterprise implementation of \gls{sip} or any other proprietary way. The message is then transported through the enterprise \gls{sbc}.


% The \gls{mcu} sends an offer via the signaling method, then on the client the remote party would install it using the setRemoteDescription() API.

% \subsection{The Transport Proxy}
% The \gls{wrtc} specification make support for \gls{ice} and \gls{srtp}-{dtls} mandatory. The problem here is that Virtual Arena uses raw \gls{rtp} streams, it does not need the added security layers that \gls{wrtc} defines. It is up to the Transport Proxy to convert the media streams to allow these two worlds to interoperate. 

% How can we take advantage of \gls{ice} and it's security? By modifying a constraint in the IceTransport object we can modify which candidates the ICE engine is allowed to use. We can indicate that the engine must use only relay candidates. This can be used to prevent leakage of IP addresses.


% \subsection{The Media Transcoder}
% The \gls{wrtc} specification defines these mandatory codecs:
% \begin{itemize}
%     \item Audio: opus and g.711
%     \item Video: ?
% \end{itemize}

% There are still discussions on the topic of which video codec should be standard. The choice is between VP8 and H.264. The H.264 codec was recently made free by Cisco, so now both choices are royalty free. H.264 is the most widely deployed and currently has the best hardware support, but both Google and Firefox has decided to use VP8 in their WebRTC implementations.

% Virtual Arena uses Speex for audio and Theora for video. So these would have to be transcoded to the appropriate formats.

% This is one of the advantages of utilizing a \gls{mcu} because you can add support for both H.264 and VP8 and be able to create a session between  both Chrome, Virtual Arena, and Bowser.


%forkalr solution


\section{Problems}


\section{Summary}
The first iteration of RTCWeb is still under development, and not all protocols and codecs have been decided yet. It is theoretically possible to create a gateway and it has been done before under closed doors. Some of the biggest issues are how to handle the signaling, translating the media, and crossing the enterprise firewall.


% \subsection{Related Work}
% There is a great interest in interworking webRTC with existing telco services. An example is the integration with IMS. This work closely resembles the work that is done in this thesis. Currently there are gateway implemenations between webrtc and ims created by ericsson and mavenir. the 3gpp is doing work on drafting a standardized gateway implementation between webrtc and ims. Both ericsson and mavenir systems are closed to th epublic, but the 3gpp papers are open for reviewing.