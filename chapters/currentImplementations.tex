
There are a lot of implementations of real-time communications out on the market today. Let's take a look at similar technologies\cite{lopez_fernandez_catalysing_2013} that is in direct competition with \gls{wrtc}:

\section{Real-time communications}
For doing real-time communications we have commercial services such as Skype, Tokbox, and Google Hangout. Actually parts of the \gls{wrtc} stack is implemented in Hangout's code base. Hangout is not a pure \gls{wrtc} implementation yet, simply beacause it needs to work on every browser, therefore it is packaged in a browser extension, but the underlying works relies heavily on \gls{wrtc} technologies.

Skype is definately the biggest one in this category. Skype utilizes peer-to-peer technology by leveraging all of the available resources in a network, this is probably why they can manage free communications. Using centralized resources is very costly. Skype has very successfully developed a video collaborations system utilizing peer-to-peer technology, but the quality depends on the users available resources. This is the same with the current networking model of \gls{wrtc}.

\section{Live broadcasting}
Services that does live broadcasting such as Ustream, Twitch, and Google Chromecast are also in competition with \gls{wrtc}. Chromecast uses \gls{wrtc} as well, but in a different way than Hangout's. It is only used for one-way streaming from you device to a TV. Such 

A lot of e-commerce sites are looking to integrate \gls{wrtc} as part of their customer service experience. Today it is normal to have live text chat, but with \gls{wrtc} it is easy to integrate video support as well.

In addition we have vendors providing technologies at different layers of the protocol stack such as Microsoft's Lync and Apple's FaceTime. It is also worth to mention common open source services like Asterisk\footnote{Telephony switching services} that are heavily used in commercial software. Asterisk has recently increased their signaling services to support \gls{wrtc} as well.

These companies have made significant investments to build their services, and with the introduction of \gls{wrtc} these companies are both threatened by competing businesses, but the tehcnology also opens up for new opportunities.

There exixst two public projects for doing server-side processing of media namely Tokbox and Lynckia. Righ now these services are limited to interact with other \gls{wrtc} endpoints. There is no interoperability with non \gls{wrtc} clients.

Missing technologies in \gls{wrtc} that we can find in other services, are PUSH capabilities, interop with other legacy systems such as SIP, more flexible group conversation architecture, advanced media sharing capabilities. These would require server-side capability.