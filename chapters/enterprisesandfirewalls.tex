\section{Enterprises and Firewalls}
Enterprise software enable multiparty conference scenarios within a guarded enterprise network. Usually with a session-centric client-server architecture. Enterprise communication software usually uses an \gls{sbc} to communicate with the outside world. \gls{wrtc} is not designed to work with current \gls{sbc} deployments. We look into how we can adapt an enterprise firewall to work with WebRTC.


\subsection{Firewall}
Enterprises use firewalls to enforce Internet Protocol access policies at the edge of their networks. These policies relate to who is allowed to access which resources. The firewall are often implemented using 5-tuple rules (source and destination IP address, source and destination ports, and transport protocol). Typically firewalls were developed to handle client-server protocols such as web browsing, email, and file transfer\cite{johnston_taking_2013}. Peer-to-peer communication systems are bigger challenges to firewalls, and since \gls{wrtc} is designed to work peer-to-peer, this introduces some problems. \gls{rtp} packets are typically described as peer-to-peer flows. Media can be sent from inside the firewall, but the reverse flow often gets blocked. This is why \gls{ice} was developed, to make firewall traversal easier. But enterprise communication software usually has a client-server architecture, to communicate with entities outside the enterprise network, it is common to use a \gls{sbc}. Currently \gls{sbc}s won't work well with \gls{wrtc}. This raises some questions:

\begin{itemize}
\item How can the enterprise firewall adapt to WebRTC?
\item How can the enterprise firewall detect and apply policy to WebRTC flows?
\end{itemize}

\subsection{Session Border Controllers}
An \gls{sbc} is essentially an application layer firewall with a signaling and media application layer gateway built in. The \gls{sbc} is usually connected in an enterprise demilitarized zone(DMZ) as a trusted enterprise network element. It blocks all unauthorized signaling and media flows, and provides a point of policy enforcement as shown in Fig\ref{?}. As specified by IETF\cite{sbc} a \gls{sbc} should support \gls{sip}, \gls{rtp}, and \gls{srtp}. \gls{sip} is a signaling protocol used for VoIP and video communications to establish RTP media sessions. By parsing SIP messages, the SBC is able to discover the transport addresses (5-tuple) to be used for the media session. The \gls{sbc} opens a filter rule permitting the \gls{rtp} traffic, and the RTP media session is able to traverse the firewall. Since SBCs are widely deployed it makes sense to reuse them for WebRTC. However, there are problems with this approach. WebRTC has no concept of sessions; instead, it has a concept of `streams'. Streams have media sources and sinks that generate and consume media flows. One possible apporach would be to translate every WebRTC session that crosses enterprise boundaries into a communication session that its existing infrastructure could handle in a session-centric manner. This could mean converting a WebRTC session into a SIP session, then converting back again.

The SBC should either be upgraded to support WebRTC or we should opt for another way of detecting WebRTC flows.

\subsection{WebRTC Firewall Traversal}
If upgrading the SBC is not trivial, we can look at the other planes for detecting an incoming WebRTC session:

\subsubsection{ICE}
There is a standard type of signaling protocol used for establishing media flows as part of WebRTC. It is built into the ICE protocol used for traversing firewalls. Before any media flows, ICE is run between two browsers. This could be used to detect an ICE exchange starting up across the enterprise border. This could then be used to distinguish a WebRTC media flow across the border, allowing policy to be applied.

\subsubsection{SRTP}
Another approach might be to use SRTP key negotiation along with \gls{dtls} to authenticate the media flow. For example, since SRTP-DTLS is used for key management, the secure edge could act as a man-in-the-middle an hence validate the public key in the fingerprint. A self-signed key from the browser could be stored in an enterprise key server. This could allow the secure edge to authenticate the browser inside the enterprise network and apply appropriate policy.

\subsubsection{Media Relay}
Another approach would be to require a media relay to be used for \gls{wrtc} media sessions crossing the enterprise boundary. There is a standard protocol for this relay, known as \gls{turn}. The enterprise firewall would be configured to block non-relayed \gls{wrtc} media flows. The enterprise would then deploy a \gls{turn} server in the \gls{dmz}, and permit media flows that go through this server. During media flow setup, the TURN server could authenticate the user and also learn what type of media flow is to be set up based on the bandwidth requested. Policy could then be applied to this media flow.

\subsection{Summary}
Doing peer-to-peer communication crossing enterprise networks are difficult. It might not be trivial to upgrade current SBC implementations, therefore we looked into other ways of detecting WebRTC traffic for applying policy to the media flows. In the next chapter we will look at the other challenges of integrating WebRTC with enterprise communication software. 