The transcoding component is simple:

\begin{itemize}
\item{How to transcode the media streams?}
\item{Which codecs do we need to support?}
\end{itemize}

There are many tools for doing this. One way is to use FFpmeg\footnote{http://www.ffmpeg.org/} like below:

\begin{lstlisting}
ffmpeg -i demo.webm -r 24 demo.mp4
\end{lstlisting}

In the simple examplea above we convert a WebM (VP8) to MP4 (H.264) with a target framerate of 24 frames per second.

When doing experiments, there was problems interoperating between different video codecs. The Media Coder in the webrtc2sip gateway seemed to be disabled, since the SDP could not agree on a similar video codec in some of the experiments.

\begin{quote}
``Please note that the Media Coder will most likely be disabled on the sipml5.org hosted server'' - http://sipml5.org/expert.htm
\end{quote}

In the Web to Jitsi experiment there was not an agreement. Jitsi only supports the H.264 video codec and Chrome the VP8 codec. It seems the media transcoder did not kick in, otherwise it probably could have worked. For the other cases, I'm not sure why video won't work, but as mentioned before I'm guessing there is a mismatch in the \gls{sdp}. Since transcoding can be done in the cloud, the quality should not really be affected by this component, with today's hardware we can do live-transcoding. Which codec used can somewhat limit latency, but the biggest latency issue has to do with the clients physical geographic location. The closer a client is to the media server the better, as the media doesn't have to travel that far.

\section{Codecs}

The \gls{wrtc} specification defines these mandatory codecs:
\begin{itemize}
    \item Audio: opus and g.711
    \item Video: ?
\end{itemize}

There are still discussions on the topic of which video codec should be standard. The choice is between VP8 and H.264. The H.264 codec was recently made free by Cisco, so now both choices are royalty free. H.264 is the most widely deployed and currently has the best hardware support, but both Google and Firefox has decided to use VP8 in their WebRTC implementations. However, the Bowser browser created by ericsson for iOS, has only implemented support for H.264 in their WebRTC solution. Our enterprise applicaton VA uses Speex for audio and Theora for video. So these would have to be transcoded to the appropriate formats. This is one of the advantages of creating a media transcoder because you can add support for both H.264 and VP8 and be able to create a session between Chrome, VA, and the Bowser browser on iOS. Problem with this component is that it's expensive in terms of processing and it may cause some delays in the stream.

\section{Summary}
The Media Coder in webrtc2sip turned out difficult to test, since it was probably turned off. However, its definately possible to do transcoding, but it is very resource demanding. We should support both VP8 and H.264 in our implementation, in addition to Speex and Theora.