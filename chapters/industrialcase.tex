%!TEX root = ../main.tex

The objective of this study is to create a model for interworking \gls{rtcweb} with an enterprise collaboration system. The collaboration is focused around audio and video communication. This model should allow the enterprise collaboration system Virtual Arena to communicate with web real-time communications. The challenge is provided by Visual Solutions\footnote{http://www.bbvisuals.com/} as \gls{wrtc} is an interesting technology, that should be explored for possible integration in the future.

\section{Target interaction goals}
When a user in VA is active in a collaborative session, any external \gls{wrtc} user should be able to authenticate himself and join the session. Since the two different systems use a totally different set of protocols and codecs, there need to be a component in between that translate all the media and messages so that the two systems can work together. The user also needs to get through the enterprise firewall, without his media traffic being blocked by a strict firewall ruleset.

The aspects of this system can be divided into different components, which then again can be divided into sub-problems:

\section{The enterprise firewall}
Crossing the enterprise firewall raises the following problem:
\begin{itemize}
\item{How can the enterprise firewall detect and apply policy to \gls{wrtc} flows?}
\end{itemize}

\section{Signaling}
The signaling component handles user authentication and signaling:
\begin{itemize}
\item{How to authenticate user identities?}
\item{How to translate the signaling?}
\item{How to handle negotiating the \gls{sdp}?}
\end{itemize}

\section{Transport}
The transport component handles the transport of media and IP addressing:
\begin{itemize}
\item{How to implement ICE?}
\item{How to negotiate the DTLS secret keys?}
\item{How to decrypt and encrypt the media streams from SRTP to RTP and reverse?}
\item{How to handle multiplexing and demultiplexing of the streams?}
\end{itemize}

\newpage
\section{Transcoding}
The transcoder will allow us to make calls between different systems using different codecs. What we need to figure out is:
\begin{itemize}
\item{How to transcode the media streams?}
\item{Which codecs do we need to support?}
\end{itemize}

\section{WebRTC mobile client}
Since RTCWeb takes care of adapting the bandwidth and other adjustments, we mainly have to think about how we do signaling. In addition, since VA allows for subscription of multiple streams of audio and video, we want to know if it's possible to receive these simultaneously on a mobile device. So, the considerations we have to think about when developing a mobile \gls{wrtc} client are:
\begin{itemize}
\item{How to handle signaling in a way that limits battery consumption?}
\item{Is a mobile phone powerful enough for doing multi-party conferencing?}
\end{itemize}