%!TEX root = ../main.tex

The objective of this study is to implement a model for interworking \gls{rtcweb} with an enterprise collaboration system. The collaboration is focused around audio and video communication. This model should allow the enterprise collaboration system to communicate with web real-time communications. This challenge is provided by Visual Solutions\footnote{http://www.bbvisuals.com/} as \gls{wrtc} is an interesting technology, that should be explored for possible integrations in the future.

\section{Visual Solutions}
Visual Solutions is a Norwegian company in the BB Visual Group\footnote{http://www.bbvisualgroup.com/} Their primary business is within the integration of operations for the oil and gas industry. They have offices in Houston, US, Brighton, UK and a headquarter in Bergen, Norway.
They contribute tailor made solutions enabling and improving collaboration and cross-disciplinary interaction across organisation units and geographic locations. One of their applications Virtual Arena\cite{solutions_b2|virtual_2014} is a powerful and interactive tool that allows IP-based application sharing within collaborative sessions. Virtual Arena supports one-to-one, one-to-many and many-to-many collaborative scenarios, with support for high-performance application sharing, audio and video communication from an advanced 3D shared scene.

In today's  world everybody owns some kind of mobile device. If a user could enter a collaborative session from the web browser of their own simple mobile device, this would be of great benefit to the customer. The user would not have to install any additonal plugins or software to their device. Many new API's are currently being developed to improve communication and hardware access in the web browsers. The \gls{w3c} has drafted a document that defines a set of API's to allow media to be sent to and from another browser or device implementing the appropriate set of real-time protocols mandated by the \gls{ietf}.

Visual Solutions wants to investigate how far development has come with these new API's and if it's possible to use them in their own applications.

\section{Target model architecture}
When a user in Virtual Arena is active in a colloborative session, any external \gls{wrtc} user should be able to authenticate himself and join the session. Since the two different systems use a totally different set of protocols and codecs, there need to be an element in between that translate all the media and messages so that the two systems can work together. The user also needs to get through the enterprise firewall, without his media traffic being blocked by a strict firewall ruleset.


\section{Existing internal system}
The existing desktop application system we use as an example for developing a gateway between \gls{wrtc} and a enterprise collaboration system is Virtual Arena. Virtual Arena's technologies can be divided into three sub compoenents:

\subsection{Signaling}
Virtual Arena uses a proprietary way of doing signaling over \gls{rtcp}. Communication between peers and the media server is done by opening up ports in the firewall to listen for incoming TCp and UDP connections. The media server can receives incoming streams and mix it in with the other streams and forward it to all the other peers. All the streams are identified using an SSRC.

\subsection{Transport layer}
Raw RTP stream over UDP.

\subsection{Media}
Speex for audio and theora for video.


\section{Technologies involved}
subproblems


The problem can be defined in two sub-problems:
\\
\\
\textbf{Problem 1}\\
Is it possible to use these new API's together with Visul Solutions current application? How can these new API's be implemented with their current software solution Virtual Arena?
\\
\\
\textbf{Problem 2}\\
It is in Visual Solutions interest to gain knowledge of how good the support for these new API's are today in desktop browsers and on mobile devices. This would give an indication to whether or not there is anything to gain from investing in further development.