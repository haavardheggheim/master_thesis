%!TEX root = ../main.tex

This thesis is written in cooperation with Visual Solutions\footnote{http://www.bbvisuals.com/}.

Visual Solutions is a Norwegian company in the BB Visual Group\footnote{http://www.bbvisualgroup.com/} Their primary business is within the integration of operations for the oil and gas industry. They have offices in Houston, US, Brighton, UK and a headquarter in Bergen, Norway.

Visual Solutions contribute tailor made solutions enabling and improving collaboration and cross-disciplinary interaction across organisation units and geographic locations.

One of their applications Virtual Arena\cite{solutions_b2|virtual_2014} is a powerful and interactive tool that allows IP-based application sharing within collaborative sessions. Virtual Arena supports one-to-one, one-to-many and many-to-many collaborative scenarios, with support for high-performance application sharing, audio and video communication from an advanced 3D shared scene.

\section{Visual Solution's vision}
In today's  world everybody owns some kind of mobile device. If a user could enter a collaborative session from the web browser of their own simple mobile device, this would be of great benefit to the customer. The user would not have to install any additonal plugins or software to their device. Many new API's are currently being developed to improve communication and hardware access in the web browsers. The \gls{w3c} has drafted a document that defines a set of API's to allow media to be sent to and from another browser or device implementing the appropriate set of real-time protocols.

Visual Solutions wants to investigate how far development has come with these new API's and if their vision can be accomplished by using them in their application.

The problem can be defined in two sub-problems:
\\
\\
\textbf{Problem 1}\\
Is it possible to use these new API's together with Visul Solutions current application? How can these new API's be implemented with their current software solution Virtual Arena?
\\
\\
\textbf{Problem 2}\\
It is in Visual Solutions interest to gain knowledge of how good the support for these new API's are today in desktop browsers and on mobile devices. This would give an indication to whether or not there is anything to gain from investing in further development.