%SUMMARIZE THE THESIS AS A WHOLE AS IN INTRODUCTION. DESCRIBE HOW EVALUATION REVEALED THAT YOUR SYSTEM IS SUCCESSFUL. DESCRIBE FUTURE WORK IN THIS AREA WHAT CONTRIBUTIONS HAVE BEEN MADE????

%INTRO
This chapter starts by discussing the the validity of the solution. Then present some thoughts about the conclusions, follwed by suggestions for future work.

With greatly increased number of smartphones and tabelts in the market, with different operating systems and browsers, it has become important to fin a unifying way to create content that works equally on across all platforms an screen dimensions.


%VALIDITY

%External validity
is concerned with whether the results of the research are applicable outside the context of this thesis. Whether they can generalized to a broader audience.

The industrial case provided
has both strengths and flaws with respect to validity

The goal was to find out how to interwork RTCWEB with Virtual Arena. Ericsson was the first to supply a webrtc gateway, and now the market is actually really crowded, however most of the implementation are based on SIP and IMS systems. There a couple of open-source gateways that i've found but they also only work with SIP systems and are poorly documented. The process of creating a gatewat is very similar to that you will find of IMS SIP implementations, even if pracically the individual steps are different. The goal remains the same.

Our solution is more extensive than the open-source solution, the open-source solutions does not take into consideration an enterprise firewall, nor

This thesis also expands on some of the problems of creating a client for mobile devices.

The guidelines derived are very general, and i have expanded upon the topic because i wanted my work to reach further than just a single industrial case. The focus was on utilizing exisitng architecture
The aim was to be pragmatic so as to be relevant for businesses, I believe the results do provide useful information that will be of aid when deciding how to approach this problem.

%Validity of the approach
A question that desverves attention is whether this solution is the best solution without having tested ona prototype. I believe this, because I frist drafted my model, than I found related work that looked greatly similar in nature. I refined my approach and expanded on a few areas.


%Summary and Conclusions
The aim of this study was to develop a model for an enterprise communication system could interoperate with RTCWEB. Two objectives were identified in order to achieve this aim.????

Further expanion on the problem. What was created. How was the solution validated.

Key takeaways

%Future Work
an actual implementation of the gateway modeled. Peraphs loooked into how to handle uuids. and expanded the model to work with datachannels as well.

Also my work will have ro refined as the IETF and W3C expands and changes things on a frequent basis.


%Closing remarks
WebRTC and rtcWeb is a very hot topic these days, it was really hard to find any relevant source materials in the beginning of my research, so it was a slow process in the beginning. Then suddenly the interest for webrtc exploded and several books came out explaining the different protocols and technologies. But Webrtc is continusly being develoeped. The interest is very high at the momen, and it seems that everyone you ask in the teclo busines is working on webrtc on way or another. Within a couple of years, the communicatins landscape may be totally different than what we see today.