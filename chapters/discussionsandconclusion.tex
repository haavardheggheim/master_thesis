%SUMMARIZE THE THESIS AS A WHOLE AS IN INTRODUCTION. DESCRIBE HOW EVALUATION REVEALED THAT YOUR SYSTEM IS SUCCESSFUL. DESCRIBE FUTURE WORK IN THIS AREA WHAT CONTRIBUTIONS HAVE BEEN MADE????

% state point and significance

%INTRO
The problem as stated in the introduction was to model a bridge between web real-time communications and enterprise communications with \gls{rtcweb} as the media stack. Many enterprises are adopting the \gls{byod} policy to the workplace, and using WebRTC technology looks to be a good solution to some of the challenges that arise with such policies. Solving this problem is important because of the greatly increased number of smartphones and tablets in the market. With different operating systems and browsers, it has become important to find a unifying way to create solutions that works equally across all platforms and screen dimensions. In this chapter I will discuss the validity of my research, conclude the results, and propose some future work.

%VALIDITY

%External validity
\section{External validity}
External validity is concerned with whether the results of the research are applicable outside the context of this thesis. While this thesis was an industrial case provided by Visual Solutions, I have tried to generalize my approach so that others can use the information to implement their own version of the solution derived from this research. Ericsson was the first to supply a \gls{rtcweb} gateway, and now the market is crowded with different vendors providing different solutions, however most of the implementations are based on SIP and IMS systems. There is the webrtc2sip open-source gateway that I've found and tested, but it only works with SIP systems and is poorly documented. The process of creating a gateway is very similar to that of the IMS-SIP implementations, even if practically the individual steps are somewhat different, the goal remains the same. My solution is a little bit more heavy on the security side than both the report from 3GPP\cite{3gpp-wrtc-access-ims} and the webrtc2sip gateway. The proposed solution in this thesis is general and reaches beyond a single industrial case. The focus was on utilizing existing architecture, and the solution aims to be pragmatic so as to be relevant for businesses. I believe the results do provide useful information that will be of aid when deciding how to approach this problem.

%Validity of the approach
\section{Validity of the approach}
A question that deserves attention is whether this solution is a good solution without having a tested prototype. I believe this, because I first drafted my model after doing extensive amounts of background research. After modeling the gateway I found related work that looked greatly similar in architecture. I refined my approach and expanded on a few areas, such as the work on mobile clients.

%Summary and Conclusions
\section{Summary and conlusions}
The aim of this study was to develop a model for a system that could interoperate an enterprise communication system with real-time web communications using \gls{rtcweb} as the media stack. This was the main objective. Four supporting problems were identified in order to achieve this aim. By answering the four supporting questions, I could achieve the main objective. The four supporting problems were different components of this gateway. In addition, I investigated issues that arise when developing a \gls{wrtc} client for mobile devices.

Contributions this work gives to the industry are to be interpreted as guidelines, since \gls{ietf} has not finished standardizing the protocols and codecs yet. There will be changes in the future that have to be taken into consideration, when developing a gateway.

There has been done similar work earlier in this area, but it's mostly proprietary or lacks documentation. The current open-source solutions do not take into consideration authentication of user identities, media multiplexing, different SDP implementations, or crossing a strict enterprise firewall. My solution expands on these issues.

%Future Work
\section{Future work}
The solution presented in this thesis is a working solution relaying on the current standards, but as the IETF finishes standardizing the protocols and codecs to be used, things may change. I also do believe there has to be some changes in regards to the way signaling is done, since \gls{sdp} doesn't really fit with what \gls{wrtc} tries to achieve, to be simple. Currently, managing the \gls{sdp} is complex and involves some unnecessary parameters to be able to work with legacy SIP systems. When the DataChannel APIs starts to mature, we will probably see more changes made by the IETF on the \gls{sdp} for the next version of \gls{rtcweb}. Further work should introduce an actual implementation of the modeled gateway. Also, I would like to see some expanded work on how to manage authentication of user identities. Perhaps look into how to handle this as a separate component. Also the model should be expanded to work with the DataChannel APIs as well. I left these out, since they are constantly changing and not very mature yet. My work will have to be refined as the IETF and W3C changes things on a frequent basis.

%Closing remarks
\section{Closing remarks}
\gls{wrtc} is a very hot topic these days. It was really hard to find any relevant source materials in the beginning of my research, so it was a slow process in the beginning. Then suddenly the interest for \gls{wrtc} exploded and several books came out explaining the different protocols and technologies. But \gls{wrtc} is continuously under development, I would definitely say that it's important for telecom companies to work on this technology, because it's a very useful tool that allows for a large variety of devices to communicate. The interest in \gls{wrtc} is very high at the moment, and it seems that everyone you ask in the telecom business is working on \gls{wrtc} in one way or another. Within a couple of years, the communication landscape may be completely different than that of today.