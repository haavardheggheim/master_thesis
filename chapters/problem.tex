%!TEX root = ../main.tex

This chapter will describe the problems of integrating \gls{wrtc} with enterprise communication systems. It will also present the goals of doing this research, present the research questions, define the research method, then lastly present the expected results.

\section{Problem description}
The \gls{byod} policy allows workers to bring their own mobile or tablet device to work. In enterprises this policy generates a lot of problems regarding how to handle security. One thing people need is to be able to access their enterprise internal communication system from their own device. The \gls{wrtc} technology allows us to easily create mobile clients, but how do we get these to interact with existing enterprise communication systems? We need some kind of technology that interworks these two technologies. We need to have the two systems being able to communicate with each other through message exchanging and signaling. We also need to have the underlying transport protocols of \gls{wrtc} work together with traditional enterprise communication protocols. In addition, the media flows may need to be transcoded. Last but probably most important is how to handle access policies and crossing the enterprise firewall. The advantages of supporting \gls{rtcweb} in enterprise communication systems are huge. Immediately the system can access any WebRTC-enabled browser. This means we can reach out to a whole range of new devices. We basically get universal mobile support, easy client integration with top security, and no need to download applications or plugins, the app runs directly in the browser.

\section{Research question}
This thesis will try to answer the following question:
\\
\\
\textbf{How can we integrate \gls{wrtc} with an enterprise collaboration system for doing communication?}
\\
\\
Supporting questions:

\begin{itemize}
\item How do we cross the enterprise firewall?
\item How can we develop a signaling proxy?
\item How can we develop a transport proxy?
\item How can we develop a transcoder?
\end{itemize}

By answering these questions, I can derive guidelines that will make it possible for multiple businesses to take advantage of the \gls{wrtc} technology. In addition I will discuss how we should approach developing a WebRTC client for mobile devices.


\section{Research method}
The research method used in this thesis is an analytical and experimental method. By looking at related work and using tools like Wireshark\footnote{Wireshark is a network protocol analyzer - https://www.wireshark.org/}, WebRTC Internals\footnote{chrome://webrtc-internals/} and Chrome Developer Tools\footnote{The Chrome Developer Tools are a set of web authoring and debugging tools built into Google Chrome}, I will carry out experiments and test the different components we need to develop, to solve the research questions. It is pretty clear by looking at the supporting questions that there is a lot of translation that needs to be done. Therefore it is logical that we create a model of a gateway that does all this translation. However, developing a \gls{wrtc} gateway is a huge amount of work, and because of all the protocols and security measures that would have to be implemented, I will not actually create a gateway in this thesis. I will derive a model describing how the theoretical gateway we create fits in with WebRTC and enterprise architecture. I will test the validity of this gateway by doing experiments using sipML5\footnote{http://sipml5.org/}, which is an open-source HTML5 SIP client using WebRTC for the media stack (natively provided by the browser). It uses SIP and SDP stacks written in javascript over WebSockets for signaling. For interoperability with native SIP clients it uses the webrtc2sip\footnote{Smart SIP and Media Gateway to connect WebRTC endpoints} gateway to act as a SIP proxy for translating the signaling. This gateway also includes a RTCWeb Breaker to convert the media streams and a Media Coder for transcoding. This gives us almost everything we need for my research, with the exception of crossing the enterprise firewall. The signaling component is a lot different from the one in VA, but this will just give an advantage in terms of deriving more general guidelines. However, I will not be able to provide specific integration guidelines for VA in terms of signaling.

\section{Expected results}

\begin{itemize}
    \item From this thesis I will have provided a model of how an enterprise can integrate WebRTC with their own communication systems. I want to provide information and guidelines about all the necessary technologies that needs to be implemented.
    \item Also I would like to provide guidelines on how to design a mobile friendly WebRTC client.
\end{itemize}
