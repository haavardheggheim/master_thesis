We live in a world where communication plays a major role in how we conduct business. Location independant work roles are more common now than ever before. Two persons no longer have to sit next to each other in order to collaborate on a project, they just need good real-time communication tools. There are several methods of doing communication, including: web-based communication, video conferencing, e-mails, telephoned meetings and forum boards. In may 2011, Google released an open-source project for browser-based real-time communication known as \gls{wrtc}\cite{google-release-of-webrtc}. There is an ongoing effort to standardise this work. The term \gls{wrtc} is usually used in general to cover all aspects of this technology, however specifically the underlying protocols and codecs are defined as \gls{rtcweb} and is standardised by the IETF\cite{ietf}. \gls{wrtc} actually only covers the browser APIs defined by the W3C\cite{w3c}. These APIs are integrated directly in the browser and allows smaller companies to create advanced communication applications in a much simpler way than ever before. These new APIs has the advantage of being able to work on every single device that runs a \gls{wrtc} compatible web browser. By integrating \gls{wrtc} into older communication systems, we allow these systems to interact with a whole range of new devices. By modeling a bridge between \gls{wrtc} and enterprise communication systems, we can reach these devices without having to modify or break current system architecture.

Huge companies like Cisco and Microsoft are big players in the communications industry. They sell tools that makes us able to communicate in real-time. These tools are expensive and generate a lot of revenue for these companies, but now we can see a rise of smaller companies competing with these big companies. These smaller companies use new web technologies that simplifies greatly the way we can develop communication applications. The new technologies are integrated in the web browser, so that we can use simple web APIs to handle communication between peers. These new web APIs are defined as \gls{wrtc}. All the advanced underlying technologies of doing transportation of media, routing the traffic through firewalls, and dynamically adjusting bandwidth usage, is taken care of by the browser. Using these new simple APIs we can quickly develop advanced communication systems, and since they are integrated directly in the browser, it is also possible to get mobile device support with a lot less effort than before. The main problem here is that these new APIs don't have any way of communicating with older communication systems. This is a problem for companies that want to take advantage of the new possibilites these APIs provide, such as mobile support for almost all platforms that can run a web browser.

In collaboration with the company Visual Solutions I will look at how we can integrate an older communication system with the new \gls{wrtc} APIs. This work could be of benefit to all communication companies that want's to take advantage of the new possibilities that \gls{wrtc} provides. Visualize yourself getting late to a meeting, with a simple click on a link you got in an invitation, you will be able to participate in the meeting from your mobile device, wherever you are with an internet connection, this would be possible. Visual Solutions have created an application called Virtual Arena that does collaboration between peers thorugh sharing of audio, video, and applications. This is the application that we will use as a reference throughout this thesis.

In this thesis we will provide a model of a solution to bridge \gls{wrtc} with a typical enterprise desktop communication system. We do this by going through all the tehnologies that \gls{wrtc} requires to be implemented for a connection between two peers to occur. Looking at different models of creating a gateway between two systems, we will provide the most feasible solution. We will also focus especially on the communication between mobile devices, as this is one of biggest advantages of implementing \gls{wrtc}.

This thesis will begin by giving an introduction to \gls{wrtc} and basic transportation protocols. To understand how \gls{wrtc} works you need to understand a lot of new concepts. These will be described in the background chapter. Then we will describe in detail what the main problems of bridging \gls{wrtc} with a desktop enterprise communication system are. We will also give a quick overview of how far the development of these new APIs have come and what is currently possible to do. We will then model a gateway that makes communication between \gls{wrtc} and Virtual Arena possible. We will also look at how this model can be improved by looking at different network architectures other than the simple peer-to-peer architecture that \gls{wrtc} utilizes. Last we will take a look into future and make a prediction of what new features will soon be available for \gls{wrtc}.