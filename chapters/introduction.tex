%!TEX root = ../main.tex

In todays business world activities are fast-paced and stretches over multiple borders. Companies often have offices located around the world. An employee can be mobile and still be able collaborate on a work project with another set of employees located in an entirely different location. All you need is an internet connection and a mobile device. This thesis will look at using new technologies for doing visual collaboration in the browser in an enterprise setting.

\gls{wrtc} support is currently under development by the biggest browser companies namely Google, Mozilla, and Opera. Several applications exists today for doing video collaboration. The biggest ones are Skype, Google Hangout and FaceTime. The \gls{wrtc} project aims to implement this technology directly in the web browser, without the need for an external application or the installation of a plugin. As of february 2013 both Mozilla Firefox and Google Chrome has integrated the new WebRTC API in their stable builds. More browsers still need to implement this new technology which is drafted by the \gls{w3c} to enable browser to browser applications for real-time communication without plugins. All current web-based video collaboration systems depend on some kind of plugin, but with this new technology we can stream both audio and video directly between browsers. The \gls{w3c} draft of the \gls{wrtc} standard is still a work in progress and there is a lot going on at the moment. Currently the technology is implemented only in desktop browsers with advanced implementations in the Chrome and Firefox browsers, but within a year it is believed that most mobile browsers will have support as well.

\section{Research Goals}
Global enterprises spend over \$2.1 trillion yearly on communication systems. Large businesses have a lot requirements that are imposed on them like being able to record a phone call handling a banking transaction. These requirements results in complex, closed, and inflexible systems that are costly to maintain and update. They often require hardware for both storage and delivery of communication. Such legacy systems hinders innovation in the communications technology field. WebRTC is a new direction that embraces new technology advancements and provides a simple and easy to use platform that is easy accessible so that developers are allowed to be more creative than before. The cost and simplicity of creating a communication system is minimal compared to older legacy systems. We can build one single application that works across multiple devices. We will se a lot of applications utilizing WebRTC in the future. With the new bring your own device policy that is taking place inside businesses comes the challenge of providing the technology and protocols for managing new scalable applications successfully.

This thesis will look at WebRTC from a business perspective, how it scales with multiple devices, it will look at different methods of balancing load and bandwidth. Look at latency and cpu usage. Also discuss the limitations of WebRTC from a business perspective. 
Will implement two demos using freely available APIs. Will also create a model by looking at cpu-load, bandwidth usage when scaling across multiple devices and multiple users. Compare the results and propose a solution. Also discuss the limitations of WebRTC.


\section{Problem Statement}
The demand for simple and cheap communication is increasing. Cisco forecasts that the number of mobile-connected devices will soon exceed the number of people on the planet. WebRTC will have a major impact on enterprise communications. This thesis will illustrate  a number of issues that are specific to WebRTC enterprise usage. Specifically relating to integration and interoperation with existing communication infrastructure. A lot of problems relate to security: firewall and peer-to-peer data flow. Currently the most typical way is to use a traditional client/server architecture. Can we take advantage of WebRTCs possibility to use P2P connections.This would in theory greatly reduce bandwidth server side.

Having identified a solution, this paper will look at existing works in the next chapter that have tried to address the problem, then conclude how a new solution would be different when implemented.

This thesis focus on the feasibility of implementing WebRTC in an enterprise environment. Attempting to aid the community by providing guidelines for scalability of native streaming in the browser using available HTML5 APIs by implementing modules and look at the hard data.


\section{BB Visuals}
The focus of this thesis is on visual collaboration systems. The company I’m working with is BB Visuals. They develop collaboration systems for the oil and gas industry. They have this idea that in the future most types of communication will take place within the browser of your mobile device. They have taken interest in a new technology drafted by the W3C called WebRTC. It is currently under development by the biggest web browser companies namely Google, Mozilla and Opera. BB Visuals want to extend their current collaboration system with a web application. Which is where I come in. I’m going to help them try and create a working prototype using this new technology. A great problem when doing visual collaboration is bandwidth saving. BB Visuals have developed something they call a “MediaHub” which has the possibility to join multiple streams into one single stream for reduced load. The end goal would be to integrate the web application with their “MediaHub”.


\section{WebRTC}
The goal of this technology is to recreate an experience that is similar to our normal way of face-to-face interaction. There exists software that does visual collaboration today. The biggest ones are Skype, Google Hangout and FaceTime. The WebRTC project aims to implement this technology directly in the web browser, without the need for an external application or the installation of a plugin. As of February 2013 both Mozilla Firefox and Google Chrome has integrated the new WebRTC API in their stable builds. More browsers still need to implement this new technology which is drafted by the World Wide Web Consortium (W3C) to enable browser to browser applications for real-time communication without plugins. All current web-based video collaboration systems depend on some kind of plugin, but with this new technology we can stream both audio and video directly in the web browser. The W3C draft of the WebRTC standard is still a work in progress and there is a lot going on at the moment. Currently the technology is implemented only in desktop browsers with advanced implementations in the Chrome and Firefox browsers, but within a year it is believed that most mobile browsers will have support as well.
