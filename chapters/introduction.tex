We live in a world where communication plays a major role in how we conduct business. Location independant work roles are more common now than ever before. Two persons no longer have to sit next to each other in order to collaborate on a project, they just need good real-time communication tools. There are several ways of doing communication, including: web-based communication, video conferencing, e-mails, telephoned meetings, and forum boards.

In may 2011, Google made available an open-source project for browser-based real-time communication known as \gls{wrtc}\cite{google-release-of-webrtc}. There is an ongoing effort to standardise this work. The term \gls{wrtc} is usually used in general to cover all aspects of this technology, like I do in this thesis. However, specifically the underlying protocols and codecs are defined as \gls{rtcweb} and is standardised by the \gls{ietf}\cite{ietf}. The strictest meaning of the definition \gls{wrtc} actually only covers the browser APIs, defined by the W3C\cite{w3c}. These APIs are integrated directly in the browser and allows smaller enterprises to create advanced communication applications in a much simpler way than ever before. These new APIs has the advantage of being able to work on every single device that runs a \gls{wrtc} compatible web browser. By integrating \gls{rtcweb} with older communication systems, we allow these systems to interact with a whole range of new devices. By modeling a bridge between \gls{wrtc} and enterprise communication systems, we can reach these devices without having to modify or break current system architecture.

Large companies like Cisco and Microsoft are big players in the communications industry. They sell tools that makes us able to communicate in real-time. These tools are expensive and generate a lot of revenue for these companies, but now we see a rise of smaller enterprises taking market shares in these areas. These enterprises are efficient at using new technologies that simplifies greatly the way we can develop communication applications. \gls{wrtc} is one of these new technologies. It is directly integrated in the web browser, so that we can use simple web APIs to handle communication between peers. All the advanced underlying technologies of doing transportation of media, routing the traffic through firewalls, and dynamically adjusting bandwidth usage, is taken care of by the browser. By using these new simple APIs, we can quickly develop advanced communication applications, and since they are integrated directly in the browser, it is also possible to get mobile device support with a lot less effort than before.

The main problem with WebRTC is that the technology does not have any way of communicating with older communication systems. This is a problem for companies that want to take advantage of the new possibilites this technology provides, such as mobile support for almost all platforms that can run a web browser.

In collaboration with the company Visual Solutions\footnote{http://www.bbvisuals.com/} I will look at how we can integrate an older communication system with the new \gls{wrtc} technology. This work could be of benefit to all communication companies that want's to take advantage of the new possibilities that \gls{wrtc} provides. Visualize yourself getting late to a meeting. With a simple click on a link you got in an invitation, you will be able to participate in the meeting from your mobile device. Wherever you are with an internet connection, this would be possible.

Visual Solutions have created an application called Virtual Arena that does collaboration between peers through sharing of audio, video, and applications. This application will be used as an example of a traditional enterprise communication system.

This thesis will model a solution to bridge \gls{wrtc} with a typical enterprise desktop communication system. I do this by going through all the tehnologies that \gls{wrtc} requires to be implemented for a connection between two peers to occur. Looking at different ways of creating a gateway between two systems, I will suggest an architecture with all the components required for making the interaction. I will also focus especially on the communication between mobile devices, as this is one of biggest advantages of utilizing \gls{wrtc}.

This thesis will begin by giving an introduction to \gls{wrtc} and basic transportation protocols. To understand how \gls{wrtc} works you need to understand a lot of new concepts. These will be described in the background chapter. Then I will describe in detail what the main problems of bridging \gls{wrtc} with a desktop enterprise communication system are. I will then model a gateway that makes communication between \gls{wrtc} and Virtual Arena possible. I will also look at different problems that needs to be handled when developing a mobile \gls{wrtc} client. Lastly, I will take a look into the future and make a prediction of what new features will soon be available for \gls{wrtc}.