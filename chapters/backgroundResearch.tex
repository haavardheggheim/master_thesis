%!TEX root = ../main.tex

This chapter will look at current video collaboration solutions and solutions that have tried video streaming utilizing P2P technology.

\section{State of Collaboration Today}
There exists millions of SIP and RTP devices that are used in Enterprises. These devices does not have a common set of features. Some are upgradeable, some are not. These devices include PBXs, desk-phones, soft-phones, PRI gateways, and voicemail servers. Some SIP devices implement ICE, while others do not even generate RTP and only support older protocols. Several software-based SIP agents (softphones) implement ICE, but almost no
PSTN gateways or PBXs do. And very few media servers do.
\\

Huge vendors in the communications business like Cisco, Polycom and Huawei initially based their business on hardware systems, but customers are now migrating over to software-based solutions and lower-cost systems. This trend is allowing new companies to come up with alternate solutions.
\\
It is difficult to require businesses to upgrade all their devices in order to communicate with WebRTC-enabled Browsers. Therefore we need an interworking function to make the systems work together. An interworking function may be needed, but it is essential to minimize the cost of such a function, or not need one to begin with. 
\\

\subsection{Security}
Another problem by utilizing the Internet and third party providers is security. The Javascript in Browser cannot be fully trusted. The current solution is to require a peer to use ICE using STUN for connectivity checks. A WebRTC browser cannot allow session media to be used unless peer uses ICE. Since many SIP devices does not support ICE, and will not be upgraded, an interworking device is needed.
\\
These kinds of problems, hinder innovation and we often end up using old legacy systems. Thankfully the next generation of tech leaders is realizing that using video makes them more productive, it helps reduce cost and plays a role in attracting the best talent available. We will see video collaboration tools on the rise in the upcoming years.


\section{Peer-to-peer technology use in the enterprise}

\subsection{Skype}
Skype utilizes P2P by leveraging all of the available resources in a network, this is probably why they can manage free communications. Using centralized resources is very costly.
Even the user directory is decentralized. While Skype has very successfully developed a video collaborations system utilizing P2P technology. The quality depends on the users available resources. A server would have resources dedicated and finetuned to this kind of communication.

\subsection{BitTorrent}
Livestream are great but they tend to go down if a lot of people wants to watch the stream at the same time. The idea is the same as with Skype., but the technology is built on top of the BitTorrent protocol. It chops up the feeds and distributes among a bunch of peers. This cuts down on lag and latency.
 utlilizes	

\subsection{Flash}
Codename Cirrus enables peer assisted netoworking using the Real Time Media Flow Protocol within the Adobe Flash Platform. Flash is the most used technology today for serving video content. YouTube Live tech?

\subsection{Silverlight}
netflix

\subsection{MCU}
A multipoint control unit (MCU) is a device used to bridge videoconferencing connections. The multipoint control unit is an endpoint on that provide the capability for three or more peers to participate in  a conference by combining streams into a singular one.
BB Visuals uses MCU!!

\section{Browser based solutions}

\subsection{Hangout}
Google+ Hangouts is an instant messaging and video platform developed by Google. It allows users to hold conversations between two or more users. The service can accessed through Google+ or through mobile apps. But the service uses a proprietary protocol developed by Vidyo, so there are for example, no free software clients for Google+ Hangouts.

\subsection{FaceTime}
FaceTime is a video telephony software application developed by Apple for the Iphone and iOS Macintosh computers. You can activate FaceTime when during a telephone call by pressing the FaceTime button. FaceTime is based on open standards:
H.264, SIP, STUN, TURN and ICE, RTP and SRTP. The service requires a client-side certificate. This protects against unauthorized access to the service controlled by Apple. This is why the service has not been implemented on other devices.
SIP ICE RTP SRTP

Vidyo Hangout FaceTime
\\
while video conferencing is the most common tool in video collaboration type services. Video collaboration can include screen-sharing, remote desktop, sharing 3d image screen, presenting, whiteboarding
\\
Nd native system. WebRTC is open-source, requires no-plugins, and allows.